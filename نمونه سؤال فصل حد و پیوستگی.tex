
\documentclass[template=tabling,81pt,headonall]{azmoon}
\usepackage{xepersian}
\usepackage{amsfonts}
\usepackage{graphicx}
\graphicspath{ {./images/} }
\settextfont{Yas}
\setdigitfont{A Iranian Sans}

\printanswers
    \teacher{آقا/خانم معلوم نیست}
    \teachertitle{دبیر}
    \city{گناباد}
    \schooltitle{دبیرستان}
    \school{نام مدرسه}
    \grade{یازدهم}
    \branch{شعبه کلاس}
    \topic{ریاضی}
    \examdate{تاریخ امتحان}
    \answertime{70 دقیقه}
    \begin{document}
	\begin{questions}
		\nointerlineskip%
		\vskip-\baselineskip
		\question{%
    برای تابع $f(x) = x + [x]$:
    \begin{parts}\part{نمودار تابع در بازه $x\in [-1 , 1)$}
\part{جدول زیر را کامل کنید.
\begin{table}
\centering
\begin{tabular}{|c|c|}
\hline
ستون ۱ & ستون ۲ \
\hline
مقدار ۱-۱ & مقدار ۱-۲ \
مقدار ۲-۱ & مقدار ۲-۲ \
مقدار ۳-۱ & مقدار ۳-۲ \
مقدار ۴-۱ & مقدار ۴-۲ \
مقدار ۵-۱ & مقدار ۵-۲ \
مقدار ۶-۱ & مقدار ۶-۲ \
مقدار ۷-۱ & مقدار ۷-۲ \
مقدار ۸-۱ & مقدار ۸-۲ \
مقدار ۹-۱ & مقدار ۹-۲ \
مقدار ۱۰-۱ & مقدار ۱۰-۲ \
\hline
\end{tabular}
\caption{یک جدول دو در ده}
\end{table}

}
\end{parts}

    }\end{questions}
    \end{document}
    