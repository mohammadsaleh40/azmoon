\documentclass[template=tabling,11pt,headonall]{azmoon}
\usepackage{xepersian}
\settextfont{Yas}
\setdigitfont{A Iranian Sans}

\printanswers

\teacher{آقای باغبان}
\teachertitle{دبیر}
\city{لارستان}
\schooltitle{دبیرستان}
\school{فارابی}
\grade{هشتم}
\branch{شهید جعفری‌نژادان}
\topic{ریاضی}
\examdate{96/12/23}
\answertime{60 دقیقه}
\begin{document}
	\begin{questions}
		\nointerlineskip%
		\vskip-\baselineskip
		
		\question[]{%
		در فرایند پیدا کردن عددهای اول بین ۲۰ و ۴۰، ب.م.م. دوومین عدد که در مضرب ۲ خط می‌خورد و ششمین عددی که از مضرب ۳ خط می‌خورد کدام است.
			\begin{fourchoice}
				\choice{۳}
				\choice{۲}
				\choice{۵}
				\choice{۷}
			\end{fourchoice}
		}
		\question[]{%
		در شکل مقابل مقدار  $O+A_1$ کدام است؟ (  $C_2=40^{O}$   ،   $C_1$  و $C_2$ متمم)
			\begin{fourchoice}
				\choice{130}
				\choice{150}
				\choice{160}
				\choice{140}
			\end{fourchoice}
		}
	\end{questions}
\end{document}