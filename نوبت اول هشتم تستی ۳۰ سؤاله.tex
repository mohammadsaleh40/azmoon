\documentclass[template=tabling,81pt,headonall]{azmoon}
\usepackage{xepersian}
\usepackage{amsfonts}
\usepackage{graphicx}
\graphicspath{ {./images/} }
\settextfont{Yas}
\setdigitfont{A Iranian Sans}

\printanswers

\teacher{آقای معلوم نیست}
\teachertitle{دبیر}
\city{مشهد}
\schooltitle{دبیرستان}
\school{معلوم نیست}
\grade{هشتم}
\branch{معلوم نیست}
\topic{ریاضی}
\examdate{به سوی بینهایت و فراتر از آن}
\answertime{70 دقیقه}
\begin{document}
	\begin{questions}
		\nointerlineskip%
		\vskip-\baselineskip
		
		\question{%
		در فرایند پیدا کردن عددهای اول بین ۲۰ و ۴۰، ب.م.م. دوومین عدد که در مضرب ۲ خط می‌خورد و ششمین عددی که از مضرب ۳ خط می‌خورد کدام است.
			\begin{fourchoice}
				\choice{۳}
				\choice{۲}
				\choice{۵}
				\choice{۷}
			\end{fourchoice}
		}
		\question{
		در شکل زیر مقدار  $O+A_1$ کدام است؟ (  $C_2=40^{O}$   ،   $C_1$  و $C_2$ متمم)
			
		\includegraphics[scale = 0.35]{image_2}
		\begin{fourchoice}
				\choice{130}
				\choice{150}
				\choice{160}
				\choice{140}
		\end{fourchoice}
		}
		\question{%
		مقدار ساده شده عبارت
		$3(2x+a)-2(3a+x)$
		در کدام گزینه آمده است.
			\begin{fourchoice}[2]
				\choice{$-8x+9a$}
				\choice{$4x$}
				\choice{$8x-9a$}
				\choice{$-8x+9a$}
			\end{fourchoice}
		}
		\question{
			مساحت مستطیل به طول 
			$(2x+y)$
			و عرض
			$(2y+3x)$
			کدام است؟
			\begin{fourchoice}[2]
				\choice{$2(x^2+y^2+xy)$}
				\choice{$4(x^2-y^2+2xy)$}
   				\choice{$2(2x^2+y^2+3xy)$}
				\choice{$4(x^2+y^2+xy)$}
			\end{fourchoice}

		}
		\question{
			مقدار 
			x
			 در معادله 
			$\dfrac{1}{2}x-\dfrac{4}{5}=\dfrac{2}{3}x$
			کدام است.
			\begin{fourchoice}
				\choice{$-\dfrac{24}{5}$}
				\choice{$\dfrac{24}{5}$}
				\choice{$\dfrac{24}{35}$}
				\choice{$-\dfrac{24}{35}$}
			\end{fourchoice}

		}
		\question{
			اگر
			$3x-3=7x-2x+5$
			مقدار
			$x+4$
			کدام است؟
			\\

			\begin{fourchoice}
				\choice{-4}
				\choice{0}
				\choice{-1}
				\choice{+1}
			\end{fourchoice}
		}
		\question{
			پنج برابر عددی منهای ۳ مساوی با سه برابر همان عدد به اضافه ۷ است آن عدد را بیابید.
			\\
			\begin{fourchoice}
				\choice{5}
				\choice{10}
				\choice{2}
				\choice{$\frac{1}{2}$}
		\end{fourchoice}
		}
		\question{
			اگر
			$b=i, a = 2i-j$
			مختصات بردار 
			$\overrightarrow{x} $
			که به صورت 
			$\overrightarrow{x}=3a+4b $
			است، کدام است؟
			\\
			
			
			\begin{fourchoice}[2]
				\choice{$6i-3j$}
				\choice{$4i+3j$}
				\choice{$i+j$}
				\choice{$10i-3j$}
		\end{fourchoice}
		}
		\question{
بردار مختصاد یک رباط طوری طراحی شده است که از مبدأ مختصات یک واحد به راست و یک واحد به بالا حرکت می‌کند و یک مکث ی‌کند و همین روند را تکرار می‌کند. اگر این رباط در مکث ششم در نقطه $C$ باشد مقدار $5C+3j$ کدام است؟
\\
		\begin{fourchoice}
			\choice{$6i+6j$}
			\choice{$30j+23j$}
			\choice{$33i+30j$}
			\choice{$20i+23j$}
	\end{fourchoice}
		}
		\question{
مقدار $x$ را طوری بدست آورید که:
\\
$2x^2-2x(x+3) = 4+7(x+5)$
\\
		\begin{fourchoice}
			\choice{$-\frac{9}{13}$}
			\choice{$\frac{5}{13}$}
			\choice{$\frac{9}{1}$}
			\choice{$-\frac{6}{5}$}
	\end{fourchoice}
		}
		\question{
		اگر یک چند ضلعی منتظم از ۶ مثلث متساوی الاضلاع تشکیل شده باشد شکل حاصل چند ضلعی است و اندازه زاویه داخلی آن چند درجه است؟
		\\
		\begin{fourchoice}[2]
			\choice{8ضلعی و 135 درجه}
			\choice{6ضلعی و 120 درجه}
			\choice{4ضلعی و 90 درجه}
			\choice{5 ضلعی و 108 درجه}
			

		\end{fourchoice}


		}
		\question{
		در شکل زیر پاره خط
		$CD$
		را از روی خطر $BE$موازی خط
		$BA$
		رسم کرده‌ایم. اندازه زاویه $\widehat{ADC}$ را بدست آورید. 

		\includegraphics[scale = 0.35]{image_12}
		
		\begin{fourchoice}
			\choice{$\frac{153}{2}$}
			\choice{153}
			\choice{108}
			\choice{$\frac{108}{2}$}
		\end{fourchoice}
		}
		\question{
			مقدار
			$x-y$
			کدام است؟

			\includegraphics[scale = 0.35]{image_13}
			\\
			\begin{fourchoice}
				\choice{60}
				\choice{120}
				\choice{100}
				\choice{140}
			\end{fourchoice}
		}
			
		\question{
			در یک ۵ ضلعی منتظم مقدار یک زاویه خارجی برابر 
		$2x+10$
			است. مقدار$x$
			کدام است؟

			\includegraphics[scale = 0.35]{image_14}
			
			\begin{fourchoice}
				\choice{62}
				\choice{31}
				\choice{41}
				\choice{81}
			\end{fourchoice}
		}
		\question{
			بردار
			$\overrightarrow{a}=3i+2j $
			و
			$\overrightarrow{b}=4i-3j $
			را داریم. بردارهای 
			$a,b,c$
			مطابق شکل زیر به ظاهر یک مثلث دیده می‌شوند. بردار 
			$\overrightarrow{c} $
			 کدام یک از گزینه‌های زیر است؟

			\includegraphics[scale = 0.35]{image_15}
			\begin{fourchoice}
				\choice{$7i-j$}
				\choice{$-7i+j$}
				\choice{$i-5j$}
				\choice{$-i+5j$}
			\end{fourchoice}
		}
		\question{
		درباره روابط بین مجموعه‌های اعداد صحیح
			$(\mathbb{Z})$ 
		گویا
		$(\mathbb{Q})$
		و طبیعی
		$(\mathbb{N})$
		کدام یک از گزینه‌های زیر برقرار است؟
		\begin{fourchoice}
				\choice{$\mathbb{N} \subseteq \mathbb{Q} \subseteq \mathbb{Z}$}
				\choice{$\mathbb{N} \subseteq \mathbb{Z} \subseteq \mathbb{Q}$}
				\choice{$\mathbb{Z} \subseteq \mathbb{N} \subseteq \mathbb{Q}$}
				\choice{$\mathbb{Q} \subseteq \mathbb{N} \subseteq \mathbb{Z}$}
			\end{fourchoice}
		}
		\question{
		حاصل کدام یک از گزینه‌های زیر گویا است؟
		
		\begin{fourchoice}
			\choice{$\frac{2\pi}{3}\times \frac{3}{4}$}
			\choice{$\frac{5\times \sqrt{3}}{\sqrt{7}}$}
			\choice{$\sqrt{\frac{4}{3} \times 2}$}
			\choice{$\sqrt{\frac{5 \times 7 \times 8}{7\times4}}$}
		
			\end{fourchoice}
		}
		\question{
		حاصل عبارت
		$\frac{4}{3} \times \frac{39}{4}$
		کدام است؟
		\begin{fourchoice}
			\choice{$13$}
			\choice{$\frac{133}{12}$}
			\choice{$\frac{43}{12}$}
			\choice{$\frac{43}{7}$}
		\end{fourchoice}
		}
		\question{
		حاصل عبارت
		$-\frac{2}{3} + \frac{6}{7}$
		چند است؟
		\begin{fourchoice}
			\choice{$\frac{4}{21}$}
			\choice{-$\frac{4}{21}$}
			\choice{$\frac{8}{10}$}
			\choice{$\frac{4}{10}$}
		\end{fourchoice}
		}
		\question{
		حاصل عبارت
		$\frac{\frac{2}{3} + \frac{4}{3}}{\frac{1}{5}}$
		برابر کدام یک از گزینه‌های زیر است؟
		\\
		\begin{fourchoice}
			\choice{$\frac{5}{2}$}
			\choice{$\frac{2}{5}$}
			\choice{$10$}
			\choice{$\frac{6}{3}$}
		\end{fourchoice}
		}
		\question{
		دانش‌آموزی در مسیر بدست آوردن حاصل عبارت
		$(-\frac{2}{7})\div (+\frac{4}{3})$
		دچار اشتباه شده و حاصل را
		$\frac{13}{28}$
		بدست آورده. راه حل این دانش‌آموز در زیر آمده است:
		\\
		$(-\frac{2}{7})\div(+\frac{4}{3}) = -(\frac{2}{7})\div(\frac{4}{3}) = -(\frac{2}{7})\times (\frac{3}{4}) = \frac{-8+21}{28} = \frac{13}{28}$
		\\
		کدام گزینه به اشتباه دانش‌آموز اشاره می‌کند؟
		\begin{fourchoice}[1]
			\choice{بعد از قسمت 
				$-(\frac{2}{7})\div (\frac{4}{3})$
				به اشتباه جای صورت و مخرج کسر
				$\frac{4}{3}$
				را جا‌به‌جا نوشته اند.}
			\choice{در ضرب دو کسر$\frac{3}{4}$ و $-\frac{2}{7}$ اشتباه شده است و به اشتباه دو کسر با هم جمع شده است.}
			\choice{علامت منفی قبل $-8+21$ فراموش شده و حاصل عبارت مثبت نوشته شده.}
			\choice{عدد ۲۱ و ۲۸ در 
			$\frac{-8+21}{28}$
			با هم ساده می‌شدند و می‌داشتیم
			$\frac{-8+3}{4}=\frac{-5}{4}$}
		\end{fourchoice}
		}
		\question{
		دانش‌آموزی در محاسبه حاصل عبارت
		$-(\frac{3}{4})\times (-\frac{8}{24})$
		اشتباه کرده و حاصل را
		$\frac{1}{4}$
		به دست آورده. مسیر حل او به شکل زیر است:
		\\
		$(-\frac{3}{4})\times(-\frac{8}{24}) = +(\frac{3}{4})\times(\frac{8}{24}) = +(\frac{3}{4})\times (\frac{1}{3}) = +(\frac{1}{4})\times 1 = \frac{1}{4}$
		\\
		کدام گزینه به اشتباه دانش‌آموز اشاره می‌کند؟
		\begin{fourchoice}[1]
			\choice{در شروع کار علامت «-» را فراموش کرده.}
			\choice{در ساده کردن عبارت $+(\frac{3}{4})\times (\frac{8}{24})$
			باید ۳ و ۲۴ را باهم ساده می‌کرد و به عبارت
			$\frac{1}{4}\times\frac{8}{8}$
			می‌رسید و بعد ادامه می‌داد.}
			\choice{ابتدا باید برای دو کسر $(\frac{3}{4})$ و 
			$\frac{8}{24}$
			مخرج مشترک ۲۴ را انتخاب می‌کرد و بعد ادامه می‌داد.}
			\choice{دانش‌آموز اشتباهی نکرده.}
		\end{fourchoice}
		}
		\question{
			تعدادی مکعب
			$1\times1\times1$
			داریم که می‌خواهیم با در کنار هم چیدن این مکعب‌ها یک مکعب مستطیل (تو پر) بدست بیاید که اضلاعش حتما بیشتر از یک باشد. مثلا مکعب مستطیل 
			$3\times4\times1$
			نمی‌تواند باشد، چون یک ضلعش «۱» است. کدام یک از گزینه‌ها می‌تواند تعداد این مکعب‌ها باشد تا بتوانیم به همچین چینشی برسیم؟
			\begin{fourchoice}
				\choice{49}
				\choice{30}
				\choice{7}
				\choice{21}
			\end{fourchoice}
			}
		\question{
		ماشین اصغر در هر ۱۰۰ کیلومتر که مسافت طی می‌کند 		
		$6.7$
		لیتر بنزین مصرف می‌کند. اصغر برای احطیات بک ۴ لیتری بنزین در ماشین دارد. در صورت خالی شدن باک ماشین اصغر تا چند کیلومتر دیگر را می‌تواند به کمک آن ۴ لیتری برود؟
		\begin{fourchoice}
					\choice{$59.70$}
					\choice{$5.97$}
					\choice{$67.50$}
					\choice{$6.75$}
				\end{fourchoice}
		}
		\question{
			در محوطه یک مجموعه آموزشی مسجد با ظاهر هشت ضلعی منتظم طراحی شده امتداد یکی از دیوارهای مسجد با ساختمان خوابگاه این مجموعه زاویه
			$15^{\circ} $
			می‌سازد. کسی داخل خوابگاه می‌خواهد رو بهه قبله بایستد. بعد از ایستادن روبه دیوار مشخص شده چند درجه باید به سمت چپ بچرخد؟
		
		
			\includegraphics[scale = 0.45]{image_25}
		\begin{fourchoice}
				\choice{$37.5$}
				\choice{$33.75$}
				\choice{$27.5$}
				\choice{$35.5$}
		\end{fourchoice}
		}
		\question{
			یک سالن ورزشی داریم که به تازگی به بخش انتهایی آن یک تشک بزرگ یکپارچه اضافه شده. ۱۶ متر انتهایی این سالن حتما باید تشک می‌شد. ولی صاحب باشگاه مقدار بیشتری از این باشگاه را تشک کرده. اگر هر متر تشک $100,000$ تومان باشد و هزینه تشک سالن $24,000,000$ تومان شده باشد. چند متر اضافه تر از طول سالن تشک شده است؟
			\\
			\includegraphics[scale = 0.45]{image_26}
		\begin{fourchoice}
				\choice{$2$}
				\choice{$4$}
				\choice{$24$}
				\choice{$48$}
		\end{fourchoice}
		}
		\question{وقتی یک قایق می‌خواهد در هر جهتی حرکت کند باید یک واحد نیرو در جهت مخالف آن جهت نیرو تولید کند تا نیروی عکس العمل آن باعث شود تا با سرعت ۱ متر بر ثانیه به سمتی که می‌خواهیم حرکت کند.
			یک قایق قصد دارد با سرعت ۲ متر بر ثانیه به سمت شرق حرکت کند. در این دریا باد شدیدی در حال وزیدن است و اگر موتور قایق خاموش شود قایق با سرعت یک متر بر ثانیه به سمت جنوب حرکت خواهد کرد. بردار نیرو موتور این قایق کدام باید باشد؟
		\begin{fourchoice}[2]
				\choice{$2i+j$}
				\choice{$2i-j$}
				\choice{$-2i +j$}
				\choice{$-2i -j$}
		\end{fourchoice}
		}
		\question{
			
		دو عدد کدام گزینه نسبت به هم اول اند؟
		\begin{fourchoice}[2]
			\choice{۱۴ و ۱۶}
			\choice{۲۴ و ۴۸}
			\choice{۹ و ۵۴}
			\choice{۲۳ و ۳۱}
	\end{fourchoice}
	}
	\question{
		در شکل زیر زاویه‌های $A$ و $B$ ۹۰ قائمه هستند. و زاویه $D$ ۳۰ درجه است. زاویه $C$ چند درجه است؟ 
			\\
			\includegraphics[scale = 0.50]{image_29}
		\begin{fourchoice}
				\choice{30}
				\choice{45}
				\choice{15}
				\choice{60}
		\end{fourchoice}
		}
		\question{
		وقتی می‌خواهیم شیئی مانند یک جعبه را روی زمین جابه‌جا کنیم نیرویی به نام اصطکاک ایستایی وجود دارد که تا وقتی اندازه نیرویی که به آن وارد می‌کنیم کم تر از آستانه نیروی اصطکاک باشد جعبه حرکت نمی‌کند. آستانه نیروی اصطکاک ایستایی جعبه‌ای 5 واحد است.		
		و فعلا یک نیرو با بردار
		$\overrightarrow{x} = 3i - 2j $
		به آن وارد می‌شود. نیروی کدام یک از گزینه‌های زیر باعث می‌شود جعبه ما شروع به حرکت کند.
		%تغییرات فرمول‌ها و گزینه‌ها نیاز هست.  
		\begin{fourchoice}[2]
			\choice{$-3i-j$}
			\choice{$-i+2j$}
			\choice{$3i + 2j$}
			\choice{$-3i + j$}
	\end{fourchoice}
	}
		
	\end{questions}
\end{document}
