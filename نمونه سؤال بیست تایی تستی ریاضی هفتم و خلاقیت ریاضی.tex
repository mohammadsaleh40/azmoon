
\documentclass[template=tabling,81pt,headonall]{azmoon}
\usepackage{xepersian}
\usepackage{amsfonts}
\usepackage{graphicx}
\graphicspath{ {./images/} }
\settextfont{Yas}
\setdigitfont{A Iranian Sans}
\usepackage{fontawesome5}

\printanswers
    \teacher{محمد صالح علی اکبری}
    \teachertitle{دبیر}
    \city{گناباد}
    \schooltitle{متوسطه دوره اول}
    \school{شاهد خاکپور}
    \grade{هفتم}
    \branch{-}
    \topic{ریاضی}
    \examdate{مهر 1402}
    \answertime{120 دقیقه}
    \begin{document}
	\begin{questions}
		\nointerlineskip%
		\vskip-\baselineskip
		\question{%
تساوی‌های زیر را کامل کنید.(به شکل عبارت توان دار بنویسید.)
    \begin{LTR}
        \begin{parts}[1]\part{$(x + y) ( x + y ) = $}
‌\\\part{$\dfrac{y \times y \times y \times y \times y}{x \times x \times x} =$}
\end{parts}
\end{LTR}
        ‌
\\
    }\end{questions}
    \end{document}
    