
\documentclass[template=tabling,81pt,headonall]{azmoon}
\usepackage{xepersian}
\usepackage{amsfonts}
\usepackage{graphicx}
\usepackage{svg}
\svgpath{ {./images/} }
\graphicspath{ {./images/} }
\settextfont{Yas}
\setdigitfont{A Iranian Sans}
\usepackage{fontawesome5}

\printanswers
    \teacher{محمد صالح علی اکبری}
    \teachertitle{دبیر}
    \city{گناباد}
    \schooltitle{متوسطه دوره دوم}
    \school{شاهد امام (ره)}
    \grade{دهم}
    \branch{۱۵۱}
    \topic{ریاضی}
    \examdate{18/11/1402}
    \answertime{90 دقیقه}
    \begin{document}
	\begin{questions}
		\nointerlineskip%
		\vskip-\baselineskip
		\section{سؤالات ساده}\question[0.5]{%
عدد 4- چند ریشه 8 ام دارد؟
    \begin{fourchoice}[4]\choice{0}
\choice{1}
\choice{2}
\choice{8}
\end{fourchoice}

            }\question[0.5]{%

		ریشه دوم عدد ۱۶ را حساب کنید.‌
\\}\question[0.5]{%
در چه مواردی ریشه گیری باعث کوچک شدن مقدار عددی عدد اولیه می‌شود؟‌
\\}\question[0.5]{%
در چه مواردی ریشه گیری باعث بزرگ شدن مقدار عددی عدد اولیه می‌شود؟‌
\\}\question[3]{%
حاصل عبارت‌های زیر را حساب کنید و تا حد امکان از زیر رادیکال خارج کنید.
    \begin{LTR}
        \begin{parts}[3]\part{$\sqrt[4]{16}\times\sqrt[4]{81} = $}
\part{$\sqrt[4]{16}\times\sqrt[2]{9} = $}
\part{$\sqrt[4]{b}\times\sqrt[4]{a} = $}
\part{$\sqrt[3]{8}\times\sqrt[3]{27} = $}
\part{$\sqrt[3]{16}\times\sqrt[3]{25} = $}
\part{$\sqrt[3]{16}\times\sqrt[3]{36} = $}
\end{parts}
\end{LTR}
        
    }\question[1]{%
توان‌های گویای زیر را به شکل رادیکالی و رادیکال‌های زیر را به شکل توان‌های گویا بنویسید. (اگر عبارتی تعریف نمی‌شود آن را بیان کنید.)
    \begin{LTR}
        \begin{parts}[2]\part{$\sqrt[7]{5^3} = $}
\part{$(4)^{\dfrac{5}{6}} = $}
\part{$(12)^{\dfrac{5}{3}} = $}
\part{$(-\dfrac{3}{7})^{\dfrac{5}{8}} = $}
\end{parts}
\end{LTR}
        
    }\question[1.5]{%
به کمک اتحاد حاصل عبارت زیر را حساب کنید. \begin {LTR} $9999^2 =$  \end{LTR}}\question[1]{%
معادله زیر را به روش مربع کامل حل کنید. \begin {LTR} $x^2+6x = 2$ \\ \\ \\ \\ \end{LTR}}\question[1]{%
معادله زیر را به روش دلتا حل کنید. \begin {LTR}$x^2+6x = 2$ \\ \\ \\ \end{LTR}}\question[1]{%
معادله زیر را به روش تجزیه به عبارت‌های جبری حل کنید. \begin {LTR}$x^2+7x + 6= 0$ \end{LTR}}\question[2]{%
عبارت‌های زیر را ساده کنید.
    \begin{LTR}
        \begin{parts}[1]\part{$\dfrac{x^3-1}{(x-1)^3}$}
\part{$\dfrac{x^6+1}{x^4+2x^2+1}$}
\end{parts}
\end{LTR}
        
    }\section{سؤالات متوسط}\question[1]{%
حاصل عبارت $\sqrt[4]{\sqrt[5]{81}}-\sqrt[5]{96}-\dfrac{3}{\sqrt[5]{81}}$ را بدست آورید.‌
\\‌
\\}\question[1]{%
اگر $\sqrt{\sqrt{3}}= \sqrt[3]{3\sqrt{x}}$ باشد مقدار x را بدست آورید.‌
\\‌
\\}\question[1]{%
اگر $\sqrt{x+2}+\sqrt{x-4} = 3$، حاصل عبارت $\sqrt{x+2}-\sqrt{x-4}$ را بدست آورید.‌
\\‌
\\‌
\\‌
\\‌
\\}\question[2]{%
عبارت‌های زیر را در صورت نیاز ابتدا گویا و سپس ساده کنید.
    \begin{LTR}
        \begin{parts}[1]\part{$\dfrac{\sqrt[3]{x}-2}{\sqrt{x+1}-3}$}
‌\\\part{$\dfrac{x^3-3x+2}{x^3+3x-4}$ \\ \begin {RTL}عبارت $x-1$ را می‌توان تجزیه کرد. \end{RTL}}
\end{parts}
\end{LTR}
        
    }\section{سؤالات باز پاسخ}\question[1]{%
تفاوت ریشه دوم و رادیکال با فرجه دو را بنویسید.‌
\\‌
\\‌
\\}\question[2]{%
یک قاعده کلی برای تعداد ریشه‌های nام یک عدد در صورت وجود بنویسید.‌
\\‌
\\}\question[0.5]{%
معادل عبارت زیر را به صورتی که علامت منفی در توان نداشته باشیم بنویسید. \
	\begin {LTR} $a^{-5}=$ \\ \end {LTR}}\end{questions}
    \end{document}
    