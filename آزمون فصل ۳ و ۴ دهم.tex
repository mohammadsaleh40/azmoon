
\documentclass[template=tabling,81pt,headonall]{azmoon}
\usepackage{xepersian}
\usepackage{amsfonts}
\usepackage{graphicx}
\graphicspath{ {./images/} }
\settextfont{Yas}
\setdigitfont{A Iranian Sans}

\printanswers
    \teacher{آقا/خانم معلوم نیست}
    \teachertitle{دبیر}
    \city{گناباد}
    \schooltitle{دبیرستان}
    \school{نام مدرسه}
    \grade{دهم}
    \branch{شعبه کلاس}
    \topic{ریاضی}
    \examdate{تاریخ امتحان}
    \answertime{70 دقیقه}
    \begin{document}
	\begin{questions}
		\nointerlineskip%
		\vskip-\baselineskip
		\question{%

		ریشه دوم عدد ۱۶ را حساب کنید.‌
\\‌
\\‌
\\}\question{%
تفاوت ریشه دوم و رادیکال با فرجه دو را بنویسید.‌
\\‌
\\‌
\\}\question{%
    عدد 4- چند ریشه 8 ام دارد؟
    \begin{fourchoice}\choice{0}
\choice{1}
\choice{2}
\choice{8}
\end{fourchoice}
    }
\question{%
در چه مواردی ریشه گیری باعث کوچک شدن مقدار عددی عدد اولیه می‌شود؟‌
\\‌
\\}\question{%
در چه مواردی ریشه گیری باعث بزرگ شدن مقدار عددی عدد اولیه می‌شود؟‌
\\‌
\\}\question{%
یک قاعده کلی برای تعداد ریشه‌های nام یک عدد در صورت وجود بنویسید.‌
\\‌
\\}\question{%
حاصل عبارت‌های زیر را حساب کنید.
    \begin{LTR}
        \begin{parts}[1]\part{$\sqrt[4]{16}\times\sqrt[4]{81} = $}
\part{$\sqrt[4]{16}\times\sqrt[2]{9} = $}
\part{$\sqrt[4]{b}\times\sqrt[4]{a} = $}
\part{$\sqrt[3]{16}\times\sqrt[3]{81} = $}
\end{parts}
\end{LTR}
        
    }\question{%
معادل عبارت زیر را به صورتی که علامت منفی در توان نداشته باشیم بنویسید. \
	\begin {LTR} $a^{-5}=$ \\ \end {LTR}}\question{%
توان‌های گویای زیر را به شکل رادیکالی و رادیکال‌های زیر را به شکل توان‌های گویا بنویسید. (اگر عبارتی تعریف نمی‌شود آن را بیان کنید.)
    \begin{LTR}
        \begin{parts}[1]\part{$\sqrt[7]{5^3} = $}
\part{$(4)^{\dfrac{5}{6}} = $}
\part{$(12)^{\dfrac{5}{3}} = $}
\part{$(-\dfrac{3}{7})^{\dfrac{5}{8}} = $}
\end{parts}
\end{LTR}
        
    }\question{%
به کمک اتحاد حاصل عبارت زیر را حساب کنید. \begin {LTR} $999^2 =$ \\ \\ \\ \end{LTR}}\question{%
اگر $\sqrt{x+2}+\sqrt{x-4} = 3$، حاصل عبارت $\sqrt{x+2}-\sqrt{x-4}$ را بدست آورید.‌
\\‌
\\‌
\\‌
\\‌
\\}\question{%
معادله زیر را به روش مربع کامل حل کنید. \begin {LTR} $x^2+6x = 2$ \\ \\ \\ \\ \\ \\ \end{LTR}}\question{%
معادله زیر را به روش دلتا حل کنید. \begin {LTR}$x^2+6x = 2$ \\ \\ \\ \\ \end{LTR}}\question{%
جدول تعیین علامت عبارت زیر را بنویسید. \begin {LTR} $x^2+6x-2$ \\ \\ \\ \\ \end{LTR}}\question{%
نا معادله‌های زیر را حل کنید.
    \begin{LTR}
        \begin{parts}[1]\part{$|\dfrac{x-1}{2}-1| \geq 0$}
\part{$|\dfrac{x-1}{2}-1| \geq 3$}
\part{$x+1 \leq 5-x < 2x +3$}
\end{parts}
\end{LTR}
        
    }\end{questions}
    \end{document}
    