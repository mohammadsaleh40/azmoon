
\documentclass[template=tabling,81pt,headonall]{azmoon}
\usepackage{xepersian}
\usepackage{amsfonts}
\usepackage{graphicx}
\graphicspath{ {./images/} }
\settextfont{Yas}
\setdigitfont{A Iranian Sans}
\usepackage{fontawesome5}

\printanswers
    \teacher{محمد صالح علی اکبری}
    \teachertitle{دبیر}
    \city{گناباد}
    \schooltitle{متوسطه دوره دوم}
    \school{شاهد امام (ره)}
    \grade{دهم}
    \branch{۱۵۱}
    \topic{ریاضی}
    \examdate{08/08/1402}
    \answertime{40 دقیقه}
    \begin{document}
	\begin{questions}
		\nointerlineskip%
		\vskip-\baselineskip
		\question[0.5]{%
خط $l$ و نقطه $A$ خارج از خط $l$ مفروض است. چند خط از این نقطه می‌گذرد که بر خط $l$ عمود است؟‌
\\}\question[1.5]{%
روش رسم خطی موازی به فاصله ۳ واحد از خطی دیگر به کمک پرگار و خط کش را توضیح دهید.‌
\\‌
\\‌
\\‌
\\}\question[2]{%
فرض کنید که برای لوزی بودن یک چهارضلعی کافی است که قطرهای آن چهارضلعی عمود منصف یک دیگر باشند. لوزی رسم کنید که طول ضلع آن 10 و ارتفاع بزرگ آن دو برابر ارتفاع کوچک آن باشد.(روش رسم با خط کش و پرگار توضیح داده شود.)‌
\\‌
\\‌
\\‌
\\‌
\\‌
\\}\question[2]{%
اثبات کنید عمود منصف‌های مثلث همرس اند.‌
\\‌
\\‌
\\‌
\\}\question[2]{%
روش پیدا کردن مرکز یک دایره که فقط یک کمان از آن در دسترس است را توضیح دهید. (خط کش و پرگار تنها ابزار در دسترس است)‌
\\‌
\\‌
\\‌
\\}\section{سؤال با نمره اضافه}\question[+4]{%
در مثلث $ABC$ عمود منصف ضلع $AB$ و $AC$ را رسم کرده‌ایم. حول نقطه $B$ دایره‌ای رسم کرده‌ایم و این دایره با شعاع $2.5$ واحد محل تقاطع دو عمود منصف رسم شده را بر روی ضلع $AB$ قطع کرده است و ضلع $BC$ را نیز در فاصله $1.5$ واحد از رأس $C$ قطع کرده است. مساحت مثلث $ABC$ را بدست آورید.‌
\\‌
\\‌
\\‌
\\‌
\\}\end{questions}
    \end{document}
    